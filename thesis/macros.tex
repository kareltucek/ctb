%%% This file contains definitions of various useful macros and environments %%%
%%% Please add more macros here instead of cluttering other files with them. %%%

%%% Minor tweaks of style

% These macros employ a little dirty trick to convince LaTeX to typeset
% chapter headings sanely, without lots of empty space above them.
% Feel free to ignore.
\makeatletter
\def\@makechapterhead#1{
  {\parindent \z@ \raggedright \normalfont
   \Huge\bfseries \thechapter. #1
   \par\nobreak
   \vskip 20\p@
}}
\def\@makeschapterhead#1{
  {\parindent \z@ \raggedright \normalfont
   \Huge\bfseries #1
   \par\nobreak
   \vskip 20\p@
}}
\makeatother

% This macro defines a chapter, which is not numbered, but is included
% in the table of contents.
\def\chapwithtoc#1{
\chapter*{#1}
\addcontentsline{toc}{chapter}{#1}
}

% Draw black "slugs" whenever a line overflows, so that we can spot it easily.
\overfullrule=1mm

%%% Macros for definitions, theorems, claims, examples, ... (requires amsthm package)

\theoremstyle{plain}
\newtheorem{thm}{Theorem}
\newtheorem{theorem}{Theorem}
\newtheorem{lemma}[thm]{Lemma}
\newtheorem{claim}[thm]{Claim}

\newtheorem{observation}{Observation}
\newtheorem{problem}{Problem}
\newtheorem{define}{Definition}
\newtheorem{alg}{Algorithm}
\newtheorem*{defstyle}{}

\theoremstyle{remark}
\newtheorem*{rem}{Remark}
\newtheorem*{cor}{Corollary}
\newtheorem*{example}{Example}

\newcommand{\mybeginfigloose}{\begin{figure}[h]}
\newcommand{\mybeginfig}{\begin{figure}[H]}
\newcommand{\myendfig}[2]{\caption{#2}\label{fig:#1}\end{figure}}

\newcommand{\mybegindef}[2]{\begin{define}[#2]\label{def:#1}}
\newcommand{\myenddef}{\end{define}}

\newcommand{\mybeginalg}[2]{\begin{alg}[#2]\label{alg:#1}}
\newcommand{\myendalg}{\end{alg}}

\newcommand{\mybeginprob}[2]{\begin{problem}[#2]\label{pro:#1}}
\newcommand{\myendprob}{\end{problem}}

\newcommand{\mybeginobs}[2]{\begin{observation}[#2]\label{obs:#1}}
\newcommand{\myendobs}{\end{observation}}

\newcommand{\mybeginclaim}[2]{\begin{claim}[#2]\label{cla:#1}}
\newcommand{\myendclaim}{\end{claim}}

%\newcommand{\myexpr}[1]{ $\begin{gathered}[t]\begin{aligned} #1 \end{aligned}\end{gathered}$ }

\newcommand{\myquote}[2]{\\ \\ \centerline{\sl{``#1''}\cite{#2}}}
\newcommand{\myloosequote}[2]{{\sl{``#1''}\cite{#2}}}

%\newcommand{\algref}[1]{\begin{alg}[#1]\end{alg}}
\newcommand{\algref}[1]{\subsubsection{#1}}

\newcommand{\myexpr}[1]{ $ #1 $ }

\newcommand{\allowpg}[1]{ \end{code}\ \begin{code} }

\newcommand{\gt}{ > }
\newcommand{\lt}{ < }


%tree items
%\usetikzlibrary{shadows}
%\newcommand{\treelist}{\tikz[overlay]\draw(-.2,-.2)--(-.2,.5) (-.2,.15)--(.1,.15);}
%\newcommand{\treeitem}{\item[\treelist]}
%\newcommand{\treeendlist}{\tikz[overlay]\draw(-.2,-.2)--(-.2,.15) --(.1,.15);}
%\newcommand{\treeend}{\item[\treeendlist]}

%\newcommand{\begincodefig}[2]{\begin{figure}[H]\caption{#2}\label{fig:#1}\begin{code}}
%\newcommand{\endcodefig}{\end{code}\end{figure}}


%moje pridavky
\newcommand{\graphplain}[1]{\begin{figure}[H]\centering\input{contents/graphs/final/#1}\end{figure}}
\newcommand{\graph}[2]{\begin{figure}[H]\centering\input{contents/graphs/final/#1}\caption{#2}\label{fig:#1}\end{figure}}
%\newcommand{\graph}[1]{\begin{figure}[H]\centering\resizebox{.9\linewidth}{!}{\input{contents/graphs/final/#1}}\end{figure}}
  \newcommand{\graphn}[2]{\begin{sidewaysfigure}\resizebox{1\linewidth}{!}{\input{contents/graphs/final/#1}}\caption{#2}\end{sidewaysfigure}}
%k\newcommand{\graph}[1]{\input{contents/graphs/final/#1}\end{figure}}
%k\resizebox{.9\linewidth}{!}{\input{plot.tex}}
\newcommand{\parspace}{\vspace{1\baselineskip}}

%%% An environment for proofs

%%% FIXME %%% \newenvironment{proof}{
%%% FIXME %%%   \par\medskip\noindent
%%% FIXME %%%   \textit{Proof}.
%%% FIXME %%% }{
%%% FIXME %%% \newline
%%% FIXME %%% \rightline{$\square$}  % or \SquareCastShadowBottomRight from bbding package
%%% FIXME %%% }

%%% An environment for typesetting of program code and input/output
%%% of programs. (Requires the fancyvrb package -- fancy verbatim.)


\DefineVerbatimEnvironment{loosecode}{Verbatim}{fontsize=\small, frame=leftline, fontshape=n, samepage=false} 
\DefineVerbatimEnvironment{code}{Verbatim}{fontsize=\small, frame=leftline, fontshape=n, samepage=true} 
\newcommand{\maybreak}[1]{\allowbreak}

\newcommand{\incode}[1]{ \\ \centerline{\texttt{#1}} \\ }

%%% The field of all real and natural numbers
\newcommand{\R}{\mathbb{R}}
\newcommand{\N}{\mathbb{N}}
\newcommand{\arrow}{\rightarrow}
\newcommand{\powerset}{\raisebox{.15\baselineskip}{\Large\ensuremath{\wp}}}

%%% Useful operators for statistics and probability
\DeclareMathOperator{\pr}{\textsf{P}}
\DeclareMathOperator{\E}{\textsf{E}\,}
\DeclareMathOperator{\var}{\textrm{var}}
\DeclareMathOperator{\sd}{\textrm{sd}}

%%% Transposition of a vector/matrix
\newcommand{\T}[1]{#1^\top}

%%% Various math goodies
\newcommand{\goto}{\rightarrow}
\newcommand{\gotop}{\stackrel{P}{\longrightarrow}}
\newcommand{\maon}[1]{o(n^{#1})}
\newcommand{\abs}[1]{\left|{#1}\right|}
\newcommand{\dint}{\int_0^\tau\!\!\int_0^\tau}
\newcommand{\isqr}[1]{\frac{1}{\sqrt{#1}}}

\newcommand{\uv}[1]{``#1''}

%%% Various table goodies
\newcommand{\pulrad}[1]{\raisebox{1.5ex}[0pt]{#1}}
\newcommand{\mc}[1]{\multicolumn{1}{c}{#1}}
