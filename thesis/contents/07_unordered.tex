Since the cause of ineffectivness of the previous proposal is the requirement stating that data should be processed in order, we will investigate the same problem without this condition. 

\parspace

If we simply dropped the condition, i.e., by letting merges merge in any order, there would arise a problem that data may get reordered differently in different branches. E.g., like in the following diagram.

\graph{ordered_conflict}

This problem may be addressed by ensuring that data series are always handled as one unit, meaning that every split will be performed on all data except data that are not used any further.

\begin{define}[sequential form of control flow]
We will say that a control flow of a flow graph $G(V,E)$ with control-flow nodes expanded and cycles removed is in \emph{sequential form} if the following conditions hold.
\begin{itemize}
  \item All inputs of $G$ are in the same component.
  \item At least one of the following conditions holds for every component $C$ of $G$: 
    \begin{itemize}
      \item $C$ does not have any input from another component of $G$.
      \item All inputs of $C$ are merge nodes such that corresponding inputs of any two input merge nodes of $C$ lead into the same partition of $G$.
      \item All inputs of $C$ are explicit queues leading from another component $D$ of $G$.
    \end{itemize}
  \item At least one of the following conditions holds for every component $C$ of $G$: 
    \begin{itemize}
      \item $C$ does not have any output going into another component of $G$.
      \item All outputs of $C$ are split nodes such that corresponding outputs of any two output split nodes of $C$ lead into the same partition of $G$ and at the same time all these splits use the same condition.
      \item All outputs of $C$ are explicit queues leading into another component $D$ of $G$.
    \end{itemize}
\end{itemize}
\end{define}

\begin{define}[unordered merge] \ \ \ 
\begin{itemize}
  \item \emph{unordered merge operation} will have the same semantic meaning as the standard merge or loop merge operation except it will not take any condition input. This operation may merge streams in an arbitrary but deterministic manner.
\end{itemize}
\end{define}

It is not clear whether every consistent flow graph can be transformed into a flow graph with control flow in sequential form. What is clear is that there exists a nontrivial class of flow graphs which are in this form.

\begin{observation}[schemas with control flow in sequential form] \ \ \ 
  Let $G(V,E)$ be a flow graph s.t. the following conditions hold.
  \begin{itemize}
    \item $G$ is consistent according to \emph{consistency of schemas} (section \ref{sec:cf_constructs}).
    \item $G$ uses only unordered versions of merge nodes in place of standard merge or loop merge nodes.
    \item All input nodes of $G$ are in preamble.
  \end {itemize} 
  Then $G$ may be transformed into a consistent flow graph with control flow in sequential form. This may be achieved by copying the schema in question onto every path between \emph{preamble} and \emph{epilogue} and joining all newly created branch or loop bodies in the means of the \emph{schema sequencing} observation. We may join these by adding any regular operation which does not alter the semantics of $G$.
  \begin{proof} 
  Sequentiality of the resulting control flow follows directly from definitions.
  \end{proof}
\end{observation}

\begin{observation}
  Both \emph{schema nesting} and \emph{schema sequencing} will produce graphs with control flow in sequential form as long as:
  \begin{itemize}
    \item All input graphs have control flow in sequential form and all paths from \emph{preambles} to \emph{epilogues} are transformed in the means of the previous observation.
    \item The condition requiring all inputs to be in the same component (w.r.t. regularity and connectedness) is not voided.
  \end {itemize} 
  \begin{proof} 
  The key observation is that the previous transformation necessarily splits \emph{preamble} from \emph{epilogue} (with respect to regularity of nodes). Thus, nesting of a new schema into a component will never create two paralel schemas except for schemas which comply with the \emph{schema sequencing}.
  \end{proof}
\end{observation}

%\begin{claim}
%Let $G(V,E)$ with $I$ and $O$ be a consistent flow graph with control flow in sequenced form. 
%\end{claim}

Consider the following problem

\begin{problem}[unordered realization of control flow]
Let $G(V,E)$ be a consistent flow~graph with $O$ and $I$ defined as in the \emph{vectorized code generation}. Let $G$ use only unordered versions of merge nodes and let the control flow of $G$ be in sequential form. Let all inputs of $G$ be in one component. Also, let control flow nodes of $G$ be expanded and let the factor graph of $G$ be acyclic. We wish to generate code which will realize graph $G$ on an arbitrary number of data series, utilizing SIMD instructions whenever possible. 
\end{problem}

\begin{claim}
We claim that the \emph{ordered crawler} (section \ref{ordered_crawler}) algorithm solves this problem as well if we make the following adjustments. In addition we claim that this algorithm will always use vectorized version of body until the termination phase is entered.
 \begin{itemize}
   \item All jumps will lead to vectorized labels except for jumps in termination handlers.
   \item If $a$ and $b$ are processing bodies of merge nodes and $C$ is a component such that both $a$ and $b$ outputs data into $C$ then $a$ and $b$ get generated into the same (algorithm) partition.
   \item All buffers have capacity at least $2*w$.
 \end{itemize}
We will again show the following points:
\begin{enumerate}
  \item Algorithm produces correct results (if it terminates).
  \item Algorithm does not enter an infinite loop during the pre-termination phase unless there is an infinite loop contained in the semantics of $G$.
  \item Algorithm terminates correctly.
\end{enumerate}
\begin{proof}[Proof of 1]
  All data series are handled as consistent packs, so every data serie is handled correctly.
\end{proof}
\begin{proof}[Proof of 2] So let there be a flow graph $G(V,E)$ and a sequence of data series such that the described algorithm enters an infinite loop which does not process any data. Let $C$ be this loop. Let $t$ be topological ordering of a factor graph of $G$. 
  \begin{itemize}
    \item Again let $M = \{v \in C \mid \neg (\exists u \in C)( u <_t v \wedge (u,v) \in C)\}$. But $M$ is empty, since every component has either all inputs sufficiently populated or none.
  \end{itemize}
\end{proof}
\begin{proof}[Proof of 3]
This proof has not changed.
\end{proof}
\end{claim}

\begin{rem} 
  The partitioning algorithm may be easily adjusted for conversion of some flow graphs into their sequential equivalents. The simplest version will handle graphs which contain simple edges which bypass some control flow schemas. This algorithm may also be further refined to handle situations like parallel branching. The point of the adustment is in choosing a 'native' outgoing control flow node which will then be pasted on any bypassing edges or before any other control flow which do not match the 'native' control flow node. One question is how to solve some implementation details in order to prevent entering of infinite loops. Another question is how to optimize the resulting graphs, since a naive implementation may produce highly inefficient results.  
\end{rem}


