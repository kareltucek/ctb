The following environments with generation capabilities are present:
\begin{description}
  \item[simple] environment was the first implemented environment. It prepares random data into arrays created according to the graph specification and runs the result for a fixed number of iterations. This environment handles only graphs consisting of regular operations.

  \item[simu] environment is an improvement of the `simple' environment. This environment was designed for testing of the provided SSE instruction table.

  \item[bobox] environment provides functionality similar to the simple environment but features interface of the bobox environment and is tailored to utilize the capabilities of aligned load and store operations. Besides the generator-related functionality, this environment also generates code which manages the envelope system of the bobox environment. However, these higher-level capabilities should have been abstracted outside of our framework and it has been a huge mistake that we have allowed pollution of our output environment by API which could have been abstracted into an unrelated unit.

  \item[cf] environment implements algorithm \ref{alg:crawler}. This environment already abstracts its API. This environment provides three functions enclosed into a class with a templated context type (this is shown in figure \ref{fig:fragmentapi}). These three functions take a number of rows to be processed and a context structure which provides arrays of data, indices and offsets. The required context structure API is shown in figure \ref{fig:contextapi}. The \ttt{\$input} and \ttt{\$output} patterns of this environment are set to \ttt{"ctx.data[\$ioindex]+ctx.index[\$ioindex]*ctx.size"}. 
\end{description}

\mybeginfig
\begin{code}
template<typename CTX>
class box_/*name*/
{
  void proc_vector_aligned   (int, CTX);
  void proc_vector_unaligned (int, CTX);
  void proc_singular         (int, CTX);
}
\end{code}
\myendfig{fragmentapi}{API of the resulting code of the `cf' environment.}


\mybeginfig
\begin{code}
struct context
{
  /???
}
\end{code}
\myendfig{contextapi}{API of the context for the `cf' environment.}
TODO: fix these apis according to the actual implementation 

