\label{sec:env2}
There are four environments with generation capabilities. We will give a detailed description only for the `Cf' environment since it is the only one which is not self-contained.

\subsubsection{Simple}
`Simple' environment was the first implemented environment. It prepares random data into arrays created according to the graph specification and runs the result for a fixed number of iterations. This environment handles only graphs consisting of regular operations.

\subsubsection{Simu}
`Simu' environment is an improvement of the `simple' environment. This environment was designed for testing of the provided SSE instruction table.

\subsubsection{Bobox}
`Bobox' environment provides functionality similar to the simple environment except for two things. First, its API is API of the Bobox environment and second, it is tailored to utilise the capabilities of aligned load and store operations. Besides the generator-related functionality, this environment also generates code which manages the envelope system of the Bobox environment. With hindsight, we have to admit that this API layer should rather have been abstracted entirely outside our framework. 

  \subsubsection{Cf}
  `Cf' environment implements Algorithm \ref{alg:crawler}. This environment already abstracts its API and fully uses the SSE instruction set.
    
    The output file consists of four functions (discussed further in Section \ref{sec:preloads}) enclosed into a class with two templated context types (this is shown in Figure \ref{fig:fragmentapi}). These four functions take a number specifying how many rows are to be processed and two context structures which provide arrays of data, indices and offsets. The required context structure API is shown in Figure \ref{fig:contextapi}. 

\mybeginfig
\begin{code}
#include "macros.h"
template<typename CTXIN, typename CTXOUT>
class box_<name>
{
  void process_single    (int, CTXIN&, CTXOUT&){...};
  void process_aligned   (int, CTXIN&, CTXOUT&){...};
  void process_unaligned (int, CTXIN&, CTXOUT&){...};
  void process_shifted   (int, CTXIN&, CTXOUT&){...};
}
#define IN_INDICES_<name> <values>
#define IN_TYPES_<name> <values>
#define IN_TYPES_IL_<name> <values>
#define OUT_INDICES_<name> <values>
#define OUT_TYPES_<name> <values>
#define OUT_TYPES_IL_<name> <values>
#ifdef RUN_BOX
  RUN_BOX(box_<name>, IN_INDICES_<name>, IN_TYPES_<name>...);
#endif

\end{code}
\myendfig{fragmentapi}{API of the resulting code of the `cf' environment.}


\mybeginfig
\begin{code}
struct context
{
  <type1>* data_0; <type2> *data_1; ...
  int index_0;     int index_1;  ...
  int offset_0;    int offset_1; ...
}
\end{code}
\myendfig{contextapi}{API of the context for the `cf' environment.}


    The alias \ttt{input}, which is meant to represent storage area in instruction patterns, is set to \ttt{"ictx.data\_\$ioindex[ictx.index\_\$ioindex+\$iindex]"}. As we have suggested earlier, the \ttt{ioindex} value is supposed to be passed as a vertex parameter of input/output nodes. The \ttt{iindex} is provided by generator's context. Indices are supposed to be incremented by the entire packsize at ends of code of partitions --- this way we do not introduce new dependencies between load instructions. Indices may be simply incremented by adding the \ttt{"\$inputinc"} pattern to the \ttt{increments} pattern layer. The alias \ttt{inputinc} is set to \ttt{"ictx.offset\_\$ioindex+=\$packsize;"}. Aliases \ttt{output} and \ttt{outputinc} are defined analogically.

    The following writer layers are used for the purpose of instruction-code generation:
    \begin{description}
      \item \ttt{global} gets outputted into the top-level scope of the processing function. Suitable for injection of testing values.
      \item \ttt{default} layer serves for the standard output.
      \item \ttt{preload} layer has the \ttt{once} flag turned on and is directed to the default layer, meaning that this layer is generated before the \ttt{default} layer while the topological ordering of all instructions is preserved.
      \item \ttt{shiftacum} and \ttt{shiftacumpreload} are analogical to the pair of \ttt{default} and \ttt{preload}. The difference is that this layer gets outputted into a different layer which is handled specially. Namely, when the \ttt{EXPANDSHIFT} alias is encountered in any output layer, the content of the accumulator is printed into a switch pattern which realises shifted loads. The resulting switch gets printed into the default layer. The alias which is used to switch different paths is \ttt{offset}. The alias which may be used in instruction patterns for the constant version of the \ttt{offset} is \ttt{constoffset}.
      \item \ttt{increments} has the \ttt{once} flag set to true and is outputted after any code of the partition. It is meant to contain index increments and therefore the load and store instructions are supposed to output either the \ttt{"\$inputinc"} or the \ttt{"\$outputinc"} pattern into this layer. 
    \end{description}

Also, there are seven macro constructs which are designed to enable easy automatic management of the resulting box. Namely these contain:
\begin{description}
  \item \ttt{IN\_INDICES\_<name>} set to coma separated list of input indices.
  \item \ttt{IN\_TYPES\_<name>} set to coma separated list of input types.
  \item \ttt{IN\_TYPES\_IL\_<name>} set to coma separated list of input types interleaved by dummy types in index order. This means that if this list is zipped with the sequence \ttt{0, 1, 2, 3...} then types correspond to their \ttt{ioindex}es.
  \item \ttt{OUT\_*} - analogical macros for outputs.
  \item \ttt{RUN\_BOX} macro expansion is performed if the macro is defined. This macro may be simply included into the resulting code fragment by means of the inclusion of the \ttt{macros.h} file. This file is, by the way, supposed to be available but is not required to contain anything specific. In that case, the macro may use the defined index and type lists to construct appropriate context types and to pass these into some processing function. Such function is then free to fill the two contexts with data and use the generated box class as it finds appropriate. 
\end{description}

We also provide a sample context template (\ttt{context.h}). This template provides storage for the required fields, basic methods for offset calculation and also accumulator lists. The lists allow generic processing of these contexts (such as preparation of input data and retrieval of the processed data). In other words, any processing environment (such as Bobox) can be simply interfaced by a single file containing a single \emph{generic} algorithm. There is no need to generate an environment-specific interface for every outputted fragment.

Regarding the context \ttt{struct}, we do realise that the creation of a separate variable per every input is awkward. The context structure may be designed in much more elegant fashion by means of variadic templates. Still, we do not feel appropriate to obscure this structure by overusing the functionality which is often seen as a subdomain of dark magic. In other words, we have decided to provide this structure for its simplicity and easily understandable interface. This decision is supported by the fact that this structure is most likely the first piece of our code which any user of our framework will come into contact with. Still, as we have mentioned above, our template does allow generic processing of data.

An arbitrary context structure may be used with the resulting code fragment, but it is bound to provide the API from Figure \ref{fig:contextapi}. If the user wishes to define his own context structure with a more elegant API (and without our API), he is welcome to do so. In this case the user will surely want to override the \ttt{input} and \ttt{output} aliases. He may do so by creating a new alias environment and by forwarding the generation call to this new environment. The necessary steps are the following ones:
\begin{enumerate}
  \item Create a new alias environment.
  \item Define its \ttt{get\_name} method.
  \item Define its \ttt{alias} method. Optionally define any aliases whose use in instruction patterns is desired. For instance, \ttt{input}, \ttt{output}, \ttt{inputinc}, \ttt{outputinc}, \ttt{offset} aliases. Note that this step effectively overrides any aliases which may be defined in any lower layer. 
  \item Define its \ttt{generate} method. Make this method forward its argument to the corresponding method of the `cf' environment. Use the new alias environment as a template argument of this call (this will apply the new aliases by specifying the new alias environment as the root environment of alias expansion).
  \item Register the new alias environment within ctb. E.g., modify the \ttt{fill} method in the \ttt{ctb.h} file or duplicate the \ttt{main.cpp} file, adding a registration call between initialisation of the ctb object and the processing command. This step may be skipped unless you want to control ctb via textual commands.
  \item Use the new environment instead of the `cf' environment.
\end{enumerate}



