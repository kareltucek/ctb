Note that in this section, we will ommit all insignificant information such as constant reference markings or template definitions.

\subsection{Loader API}

Loaders are required to implement the following five methods:

\begin{code}
void load_graph(G& graph, istream&);
void load_instab(IT& instab, istream&);
void export_graph(G& instab, ostream&);
void export_instab(IT& instab, ostream&);
static string get_name();
\end{code}

\begin{description}
\item \ttt{G, IT} are templated instruction table and generator types.
\item \ttt{istream, ostream} are C++ implementation of input and output streams. This way, direct interfacing with another programme is possible.
\item \ttt{get\_name} is required to return a sensible name not colliding with names of other loaders.
\end{description}

We believe that provided the description above, the api is self describing.

When a new loader is implemented, it is supposed to be registered within the ctb object by its templated \ttt{register\_loader} method. This call may be inserted into the \ttt{fill} method of ctb. Or may be inserted into a new instance of the \ttt{main.cpp} file. Registration is not neccessary if the user does not wish to use the text command controls.

\subsection{Aliasenv API}

Fully functional alias environment which acts as an output environment is supposed to provide the following interface:

\begin{code}
typedef <typename> language;
static string alias(string a, bool* s = NULL);
static string get_name();
static writer generate(int packsize,  G& generator, 
  string name, stringlist args);
\end{code}

\begin{description}
\item \ttt{language} is a language structure which is to be used by any writer parametrized by this environment.
\item \ttt{alias} is a method which tries to evaluate the string a. If the second parameter is not null, it is also supposed to report whether the attempt was successfull.
\item \ttt{get\_name} is again any sensible string identifier.
\item \ttt{generate}
  \begin{description}
    \item \ttt{packsize} is the size of the pack for which the body is supposed to be generated, i.e., the $w$ from chapters dedicated to theoretical analysis. The environment is allowed to ignore this parameter and decide its own suitable size.
    \item \ttt{generator} is again a templated generator type. This variable provides direct access to the underlying graph as a simple member variable.
    \item \ttt{name} is a simple identifier meant for identification of the generated code fragment. E.g., the alias environment may output its results as member methods of a class whose name contains this identifier.
    \item \ttt{args} is a list which may contain other arguments.
  \end{description}
\end{description}

If an environment is constructed to act just as an alias processor, only the \ttt{alias} method and the \ttt{language} type are obligatory. If an environment is not supposed to be used as the top-level parameter of writer, the \ttt{language} field may be omitted as well.

As other structures, enviroments which provide the \ttt{generator} method are supposed to be registered within ctb by the \ttt{register\_aliasenv} method.

\subsection{Generator API}

Generator provides the following methods for the purpose of code generation. 

\begin{code}
void generate(int packsize, 
    imp_contB<W>& w, imp_contB<output_options>& options); 
void generate_partition(int partition, int packsize, 
    imp_contB<W>& w, imp_contB<output_options>& options);
imp_contB<output_options> option_struct();
int partition_count();
bool partition_is_topo_max(int partition);
bool partition_is_topo_min(int partition);
void update_types();
void update_factor();
graph_t& get_factor();
\end{code}

\begin{description}
\item \ttt{w} is the writer which is supposed to receive the output. This type may be constructed as follows:
\begin{code}
impl_contB<writer<example_aliasenv>> w;
\end{code}
 The results in nondefault layers are to be retrieved as follows:
\begin{code}
w["empty_output_indicators"].write_str();
\end{code}
\item \ttt{update\_types} forces full type inference and structure check. This may be neccessary during graph transformations since.
\item \ttt{update\_factor} updates the factor graph. This has to be called manually because unstable factor graph is not desired during transformations.
\item \ttt{options} specify special behaviour for specific layers. Instance of this structure may be either manually constructed or obtained by the \ttt{option\_struct} method. Setting options works exactly as in case of the writer. For instance:
\begin{code}
auto o = generator.option_struct();
o["empty_output_indicators"].unique = true;
\end{code}
Currently available options are:
\begin{description}
\item \ttt{once} causes generator to write an instruction into such layer only once per vertex.
\item \ttt{global\_once} causes generator to write the value into such layer only if the corresponding layer of writer does not contain anything yet.
\item \ttt{layer} causes generator to write the pattern into the layer specified by this field. This applies only if this field is non-empty.
\item \ttt{order} allows specification of evaluation order of patterns. Patterns of layers with lower \ttt{order} field get outputted before those with higher \ttt{order}. This defaults to \ttt{-1}. This is useful if combined with the \ttt{layer} option since this allows the user to specify how patterns of different fields should be interleaved. 
\end{description}
\end{description}

For the purpose of graph transforms, the generator provides the following methods. These methods interface the corresponding methods of the member graph object while handling generation-related data specially. Advanced functionality, such as custom graph crawling is provided by the graph itself.
\begin{code}
node* get_vert(J id);
node* add_vert(v_id v, op_id op, map<string,string> params);
void add_edge(v_id from, v_id to, int to_pos, int from_pos, int layer = 0);
void rm_vert(J v);
void rm_edge(edge* e);
void connect_as(J v, K as, bool inputs = true, bool outputs = true);
void foreach(function<void(node*)> f);
\end{code}

\begin{description}
  \item \ttt{J, K} are automatically resolved types identifying vertices. These may be pointers or references to the actual vertex structure or just vertex ids.
  \item \ttt{v\_id, to\_pos, from\_pos, params} correspond to the described field semantics.
  \item \ttt{get\_vert, add\_vert, rm\_vert, add\_edge, rm\_edge} apparently allow altering of the graph structure.
  \item \ttt{connect\_as} adds new edges to the vertex \ttt{v} imitating the way the \ttt{as} node is connected.
  \item \ttt{foreach} calls the function \ttt{f} once per every vertex in topological order. This function may be nested.
\end{description}

The retrieved nodes then provide the following api:
\begin{code}
T data; I id;
int classid, colourmark; 
imp_cont<vector<edge*>> in, out;
edge* in_at,out_at(int i, bool check_uniqueness = true);
\end{code}

\begin{description}
  \item \ttt{id} contains vertex id.
  \item \ttt{classid} is id of the component of connectedness if the factor graph is up-to-date. Make sure you have called the \ttt{graph.update\_factor()} method. This is not done automatically since factor graph and this id is apparently not stable with respect to factorisation.
  \item \ttt{colourmark} is and auxiliart identifier which may be used by the user for marking of vertices during graph algorithms.
  \item \ttt{data} contain the data type specified by the \ttt{T} template parameter. The generator's data type contains a pointer to the corresponding operation (\ttt{op}), an operation id (\ttt{opid}), parameter access methods (\ttt{get\_param(string)} and \ttt{set\_param(string,string)}). 
  \item \ttt{in, out} are containers containing pointers to incoming and outgoing edges. Layer 0 may be iterated by simple \ttt{for(e : in)}, other layers by \ttt{for(e : in.get\_layer(1))}.
  \item \ttt{in\_at, out\_at} provide direct access to the \ttt{i}th edge.
\end{description}

The underlying graph may be accessed directly through the \ttt{graph} member. The factor graph may be accessed through the \ttt{factor} member of \ttt{graph}. The factor graph again provides the same api as shown above. Namely, the \ttt{get\_vert(int p)} method returns a vertex representing partition \ttt{p}. Its \ttt{data} member contains a set \ttt{vertices} of vertices of this component. Furtermore there are sets \ttt{in} and \ttt{out} of vertices which connect this component from/to another partition.  

\subsection{Instruction table import API}

Instruction tables may be constructed via the following api:

\begin{code}
operation& add_operation(opid_t op_id, tid_t out_type, 
  vector<tid_t>& in_types, flag_t flags);
void operation::add_code(vector<int> widths_in, int width_out, string code, 
  stringlist code_custom, string note, string tags, int rating);

type& add_type(typename T::tid_t t, int bitwidth = 0);
void type::add_code_type(int width, string code, string note);
void type::add_code_conversion(int width_in, int width_out, 
  string code1, string code2, string code_custom, 
  string code_generic, string note, string tags, int rating);

void add_expansion(opid_t op_id, string name, string transformer, 
  vector<opid_t> args, string note, vector<tid_t> in_types, 
  vector<tid_t> out_types);
\end{code}

\begin{description}
  \item \ttt{add\_operation, add\_type} look up the requested operation or type. If the object does not exist, it is created. Finally reference to this object is returned, allowing insertion of child elements.
  \item other fields are exactly the fields described by field semantics.
\end{description}

\subsection{Graph transformer API}

Api required from custom transformers is the following:
\begin{code}
 static string get_name();
 void transform(G& generator, stringlist args);
\end{code}

Semantics of these fields are again the same. There are no restrictions on content of \ttt{args} and no restrictions on what the transformer is allowed to do with the generator or with its graph.

\subsection{Tag handler API}

Tag handlers are supposed to provide a single function:
\begin{code}
bool are_satisfied(string tag_list);
\end{code}

The \ttt{tag\_list} is a string list of tags of an instruction or width conversion. The return value is boolean which tels whether the tags satisfy \emph{some} conditions. The tag structures may be defined by whoever wants to define them. Currently, we use one tag structure to pass command-line specified tags and another one to restrict instruction sets for the `bobox' environment. Tag handlers are passed directly to the instruction table via a shared pointer. This is the only class in our framework which needs to use dynamic polymorphism.

\subsection{Custom commands}

Custom commands may be registered within the ctb class by the following method:
\begin{code}
void register_command( string cmd, function<void(stringlist&&)> f,
  string description);
\end{code}

\begin{description}
\item \ttt{cmd} -- the command string.
\item \ttt{f} -- function performing the command.
\item \ttt{description} -- short reference which will by shown in the '-h' option.
\end{description}

  


