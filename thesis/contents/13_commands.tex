
The input programme has to conform to the following rules:

\begin{itemize}
  \item Every line represents exactly one whole command unless this line is empty or commented.
  \item Any line can be commented by insertion of \ttt{\#} at its beginning.
  \item Command consists of non-empty space-separated strings. Every sequence of one or more spaces is considered as one delimiter.
  \item The first string in the list identifies the command to be run. The rest of the list is passed as arguments to the command.
\end{itemize}

The available commands are:

\begin{itemize}
  \item \ttt{help} shows a standard unix-like help.
  \item \ttt{source <input file>} -- interprets contents of the file as its argument.
  \item Import and export commands. These take identifier of a loader class as the first argument and a filename as the second argument. The structures are loaded to/exported from the same object of a graph or of an instruction table. The structure is always cleared before load. Note that not all loaders have to implement all import end export functions. 
  \begin{itemize}
    \item \ttt{exportgraph  <loader> <output file>} -- available for the `dot' loader
    \item \ttt{exportinstab <loader> <output file>} -- available for the `csv' loader
    \item \ttt{loadgraph    <loader> <input file>} -- available for the `xml' loader
    \item \ttt{loadinstab   <loader> <input file>} -- available for the `xml' and `csv' loaders
  \end{itemize}
  \item \ttt{generate <output aliasenv> <output file>} -- the \ttt{aliasenv} is invoked with the loaded generator and instruction table. The produced string is written into the \ttt{output file}.
  \item \ttt{testgraph} -- testgraph is a custom command which acts as a special version of loadgraph. This loader function constructs a graph from the loaded instructions, trying to utilise all available operations. 
  \item \ttt{adddebug  [depth=1 [<list of vertex ids>]]} -- This command hooks a debug node\footnote{These are special output nodes which are supposed to output values directly. These are not subject to width conversions.} to every vertex specified in the space separated list and to all its predeccesors up to \ttt{depth}. Debug nodes are automatically selected from operations of type \emph{debug}. If the list is empty, debug nodes are hooked to all vertices.
  \item \ttt{transform  <transformer>} -- the graph transformed identified by the first argument is invoked. This transformer is free to edit the loaded graph as it pleases.
  \item \ttt{visualise} -- writes the currently loaded graph into a temporary file and invokes the dot (graphviz) with gpicview to show the graph to the user. We use this mechanism for error reporting as well.
  \item \ttt{preprocess <aliasenv> <filein> <fileout> } -- processes the \ttt{filein} by the aliasenv identified by \ttt{aliasenv} and writes the results to \ttt{fileout}. The \ttt{<aliasenv>} string may be found in a list provided by the \ttt{ctb -h} help.
\end{itemize}
