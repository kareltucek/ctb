At this point we would like to examine possibilities of realization of flow graphs with all 4 types of nodes. 

\parspace

In the previous section we have shown how pipelines composed of regular operations may be processed. This approach does not suffice since different nodes may need to process data at different speeds. 

\parspace

\FloatBarrier

Our strategy is to cut the flow graph into partitions such that every partition consists only of regular operations. Then we add actual buffers between these partitions. The data will then be processed by a crawler which will go through our graph and process its partitions one at a time. Every single partition will be processed by code generated by generators we have already presented. 

\parspace

First, we will need some definitions.

\begin{define}[edge layer]
Let there be a graph (possibly directed multigraph) $G(V,E)$. We will assign one more numeric annotation to every edge. We will call this annotation \emph{layer}. Thus we have $layer: E \arrow \N_0$. If $layer(e) = n$ we will say that $e$ is on layer $n$.
\end{define}

Layer will denote type of edge. It will allow us to talk about subsets of edges in a more intuitive manner and also to introduce a more complex graph structure. We show our visual convention in the following picture.

\graph{lines}

\begin{define}[factor graph]
  Let $G(V,E)$ and $G_f(V_f,E_f)$ be graphs (possibly directed multigraphs). We will say that $G_f$ is a \emph{factor graph} of $G$ if there exists a function $f: V \arrow V_f$ such that:
\begin{itemize}
  \item $f(u) = f(v)$ if and only if there exists an undirected path $P_{u,v}$  in $G$ which uses only edges on layer 0. 
  \item $\exists(e =: (u,v) \in E)(layer(e) = 1) \Longleftrightarrow \exists(e_f \in E_f)( e_f = (f(u), f(v)) )$
\end{itemize}
\end{define}

Note that this definition overrides another meaning of this term.




