Reader should be familiar with the~basic terminology of the~graph~theory. For now we will assume knowledge of the~following terms: \emph{directed~graph}, \emph{multigraph}, \emph{topological~ordering}, \emph{path}, \emph{cycle}, \emph{connected component}, \emph{depth first search (DFS}. Definitions of these may be found in \cite{kapitoly}. Reader should also be familiar with some basic abbreviation of instruction names and should understand basic concepts of CPU. We will assume knowledge at least of the following abbreviations: \emph{LD (load)}, \emph{ST (store)}, \emph{ADD(+)}, \emph{SIMD (single instruction processing multiple data)}. We will also provide some definitions of our own.

\begin{define}[operation and instruction]
We will use the~term \emph{operation} for general identification of an~operation which is to be performed on some data. The~term \emph{instruction} will be used for a~specific way of performing an~operation. This way, multiple instructions may be associated with a~single operation.
\end{define}

E.g., multiplication by a constant power of two is a~unary operation. Associated instructions may be:
\begin{itemize}
  \item binary shift to left
  \item multiplication using the~asterisk operator
  \item a~vectorized version of multiplication
\end{itemize}

\begin{define}[width of an instruction]
We will use the~term \emph{width} to denote the~number of independent data series a~single instruction operates~on.
\end{define}

\begin{define}[flow graph]
We will use the~term \emph{flow~graph} for any tuple $(G(V,E),O,from:E \arrow \N, to:E \arrow \N, op:V \arrow O)$ which satisfies the~following conditions:
\begin{itemize}
  \item $G(V,E)$ is a~nonempty connected directed acyclic multigraph with annotations defined by $from$, $to$ and $op$.
  \item $O$ is a~set of operations.
  %\item $\forall(v) \in V |out(v)| \leq 1$ 
  \item Any directed edge is uniquely identified by its destination vertex and the~$to$~annotation. I.e.:
    $$ \forall{e=:(a,b),f=:(c,d)} \in {E(G)}: b = d \wedge to(b) = to(d) \Rightarrow e = f $$
\end {itemize}
    We will write $G(V,E)$ for brievity.
\end{define}

This definition of flow graph is almost identical to the definition of \emph{gate networks} given in \cite{ads}, although our use of flow graphs will resemble more the~use of \emph{kahn networks}.

\begin{define}[realization of a flow graph]
  Let $G(V,E)$ be a~flow graph and ${In = \{v \in V \mid |in(v)| = 0\}}$. \emph{Realization~of~a~flow~graph} will be an~algorithm which provided some annotations of outgoing edges of vertices in $In$ produces annotations of all other edges of $G$ such that semantics assigned to $O$ are preserved. 
\end{define}

  Flow graphs will represent pipeline computations. Edges will represent data queues. Vertices will represent data transformations. A~transformation on a \emph{regular} vertex~$v$ will pop exactly one element from every incoming queue of $v$ and push exactly one element into every outgoing queue of $v$. Element pushed into an~outgoing queue~$e$~of~$v$ will be determined from the~elements taken from the~incoming queues~of~$v$, their~$to$~annotations, the~annotation~$op(v)$ and the~annotation~$from(e)$. The~$to$~annotation identifies arguments for operations represented by members~of~$O$. The~$to$~annotation identifies outputs of operations which return more than one value.

\parspace

Note that this scheme describes semantics of every single data serie separately. 

\parspace

Also, note that our definition~of~flow~graph does neither~require nor~ensure semantics~of~$O$ to be consistent. Thus, some flow~graphs may have no realization. We will implicitly assume our flow~graphs to~be consistent with some~semantics assigned~to~$O$. This semantics will always be~clear from their context.

