\label{sec:preloads}
Most of our implementation assumes to receive its input by means of more-or-less continuous arrays (e.g., by a sequence of long arrays per every input). The SSE extension provides fast aligned load and store operations whose use is indeed desirable for this purpose. The aligned load/store operations load vectors of values from continuous memory areas which are required to begin at multiples of 16 bytes (128 bits). 

So, ideally, we would like all these arrays aligned to 16 bytes and be of lengths of multiples of 16 bytes. However, the world is not always ideal, so we dedicate this section to alternative approaches. Besides the fast loads, the extension also provides the \ttt{alignr} operation. This operation takes two SSE registers and an \emph{offset}, concatenates the two registers and selects a new (continuous) SSE register shifted by \emph{offset}. 

The \ttt{alignr} operation allows implementation of an efficient non-aligned load. This works by preloading the first aligned area into an auxiliary variable and then chaining a sequence of load operations through the \ttt{alignr} operation (as shown in Figure \ref{fig:alignedloads}). One problem is still present --- the last argument of \ttt{alignr} has to be a compile-time constant, which means that either all offsets have to be known or that we have to switch over all possible offsets. We implement the latter option per every operation. Stores may be implemented analogically.

\mybeginfig
\begin{code}
_m128 aux1, aux2;
_m128 aux1 = load(input[j-offset]);

aux2 = load(input[j-offset+1*packsize]);
_m128 a1 = alignr(aux2, aux1, offset);
aux1 = aux2;

aux2 = load(input[j-offset+2*packsize]);
_m128 a2 = alignr(aux2, aux1, offset);
aux1 = aux2;

...
\end{code}
\myendfig{alignedloads}{Example of an efficient non-aligned load of sequence of SSE registers.}
\begin{description}
  \item[input] is an array of input data.
  \item[j] represents the first index of the currently processed pack of data.
  \item[offset] is the difference between the actual address of the required data and the preceding aligned address.
\end{description}

As the result, we propose generation of four different instances of the graph-processing code fragment (we have already shown these in Section \ref{sec:env2}). Use of these code fragments would then be switched during the processing in order to achieve an efficient processing of all the data. These fragments would be:
\begin{itemize}
\item Non-vectorised realisation.
\item Vectorised realisation utilising unaligned load and store operations.
\item Vectorised realisation utilising aligned load and store operations.
\item Vectorised realisation utilising shifted load and store operations.
\end{itemize}

The vectorised realisations would be used on sequences of data rows such that all input and output arrays are continuous across these sequences. Aligned version would be used if possible, shifted or unaligned version would be used otherwise. 

The non-vectorised version could be used to process preambles and epilogues around transitions of data containers. This means processing the last \ttt{offset} rows before an end of any of the containers as well as processing the first \ttt{packsize - offset} after beginning of any container. The former processes the last \ttt{offset} lines which cannot be loaded by the vectorised algorithm. The latter ensures that the rows processed by vectorised algorithm stay aligned with output containers. However, this depends on the actual algorithm - the 'bobox', 'simple' and 'simu' environment do need this (explicit) processing of preambles and epilogues but the 'cf' environment (Algorithm \ref{alg:crawler}) does not, as long as all alignment-related conditions are satisfied.


