Our framework was developed under Linux and therefore it assumes standard Unix tools for development. The source files themselves are supposed to be platform-independent. 

\begin{itemize}
  \item C++11 compliant compiler. Tested compilers are GCC (4.9.3) and CLang (3.7.1).
  \item TinyXML2 library 
  \item Make (optional - for compilation of executables, testing, development)
  \item Dot, gpicview (optional - for graph visualisation)
  \item Standard shell tools (optional - for test scenarios and development) 
  \item Functional bobox environment with aligned memory allocator (optional - for scenario 6)
  \item CPU implementing SSE 4.2  (optional - for test scenarios which generate SSE instructions)
\end{itemize}

The project may be compiled by make:
\begin{code}
> make
\end{code}

By default, make compiles the ctb executable and unit tests and runs basic test scenarios. The following make targets are available:

\begin{description}
  \item[basictest] compiles and runs basic regress tests.
  \item[fulltest] runs full check of the sse instruction table and a test of bobox environment.
  \item[ctb] compiles the ctb executable.
  \item[all] defaults to `ctb' and `basictest'
\end{description}


This library may be used in three ways:
\begin{itemize}
  \item As a separate executable controlled by text commands.
  \item As a header C++11 library. In this case, the \ttt{"ctb.h"} file has to be included in the project. The \ttt{ctb::ctb} class serves as the library frontend. This frontend may be used in two ways, either of them requiring creation of a single object of the ctb instance:
  \begin{itemize}
    \item using text commands
    \item using raw templated calls to commands
  \end{itemize}
\end{itemize}

Basic usage reference of the provided executable may be obtained by:
\begin{code}
> ctb -h
\end{code}


The text command controls of this library may be very easily interfaced by raw C commands, which should allow direct language-independent integration.

The simplest way to invoke ctb is by:
\begin{code}
> ctb -f programme
\end{code}

Here \ttt{programme} is a text file containing a sequence of commands.





