The out-of-order solution seems to be superior to the ordered one in many aspects. An optimal situation would, of course, be if we could return data rows in an arbitrary order. Unfortunately, the data receiver may need the data ordered.  We, of course, may add row indices into the data rows for identification, but that does not solve the problem entirely, since either we will have to store the results one by one or someone else will have to retrieve them one by one. Either version would increase the amount of data transfers between CPU and RAM significantly. 


Ideally, we would like to reorder the data back using as few RAM stores and loads as possible. For this purpose, we may use the information about performed shuffling of the data. We propose the following approach:
\begin{enumerate}
  \item We process all the data and store the results into the RAM. Also, we store a sequence of Boolean values per every split or merge node. This sequence will contain a single value per every data row which has been processed by the node in question. This value will represent either left or right branch of the split or merge.
  \item We construct a new pipeline which will be an exact copy of the processing pipeline but will be built backwards and will not contain any data processing.
  \item Then we pipe the data from RAM into the new pipeline in inverse order. Also, we provide every split and merge node with the values stored by the corresponding merge or split as a control.
\end{enumerate}

\begin{rem}
  Note that we cannot pipe the original pipeline directly to the one built backwards. Imagine that the first data row to enter the processing pipeline is the last one to come out. If we piped the first pipeline directly into the sorting one, we would have to store \emph{all} the data in CPU registers until the first data row was processed and could be piped into the sorting pipeline.
\end{rem}

This approach surely has its drawbacks, but it may, for some graphs, reduce the amount of RAM transfers needed for data reordering.
