\begin{define}[width conversion]
  We will use the~term \emph{width~conversion} for an~instruction which takes data from another instruction of~some~width and transforms them to data suitable for an~instruction of a~different width. 
\end{define}

Consider the~following problem.

\begin{problem}[vectorized code generation]
  Let $G(V,E)$ and $O$ be defined as in the~\emph{simple~code~generation} problem. Furthermore, let there be a~set~$I$ (of SIMD instructions) and a~partial~function $enc (operation, width): O \times \N \arrow I$. Let $enc$ be defined at~least~on $O \times \{1\}$. Also, let there be a~set of~width conversions sufficient for direct conversion between any two widths present in $I$. Finally, let all widths be powers~of~2. We wish to~generate C~code which will realize the~graph~G using instructions of highest possible widths. I.e., every vertex will be performed by the~broadest instruction its operation is associated with. 
\end{problem}

If highest instruction widths of all operations present in~$G$ were the same ones, we could use the~\emph{simple~generator}. We would simply replace all types by the corresponding vector versions and use vectorized coutnerparts of all instructions. Typically, the~width will vary with the~bit~length of the~data~type in question and thus we cannot use this approach with more than one data type present in~$G$.

\parspace

We present different approach. First, we will find maximum width~$w$ of an~instruction which may be used in~$G$ (we will not consider operations which are not in~$G$ at~all). Then we will generate code processing $w$ data series in one iteration. Again, we take vertices of~$G$ in~topological order and process them one by one. For a~vertex whose broadest instructions is of~width~$n$ we will generate $w/n$ instances of its broadest instruction. Arguments are again retrieved from the~incoming vertices, only now we remember a~set of~names for every width that was generated. I.e., we look whether there is a~set of~names for~width~$n$. If there is, then we use it. If there is no such set, we will use some width conversion to obtain data for width~$n$. 

\parspace

Thus, pseudo code for vectorized generation may be:

\begin{samepage}
\begin{code}
fun gen_instruction(instruction, result_name, arg_1, arg_2, ...) 
  { generate actual text representation from arguments; }
fun gen_conversion_ins(from_width, to_width, result_names, arguments) 
  { generate actual text representation from arguments; }
fun vertex::in(index) 
  { return u such that there exists e = (u, v) and to(e) == index; } 
fun vertex::results(w) 
  { return array of names for width w } 

fun gen_conversion(v, to_width)
{
  from_width = native width of v;
  for(int i = 0; i < w; i+= max(from_width,to_width))
  {
    result_names, arguments;
    for(int j = 0; j < from_width/to_width || j == 0; ++j)
      result_names.push("$v_conversion_w${to_width}_${i+j}");
    for(int j = 0; j < to_width/from_width || j == 0; ++j)
      arguments.push(v.results(from_width)[i+j]);
    gen_conversion_ins(from_width, to_width, result_names, arguments);
    v.results(n).push(result_name);
  }
}

fun gen_vertex(v)
{
  for(vertex u : incoming to v)
    if( u does not have results for width n )
      gen_conversion(u, n);
  for(int i = 0; i < w/n; ++i)
  {
    result_name = "$v_$i";
    gen_instruction(enc(op(v), n), result_name, 
        v.in(1).results(n)[i], v.in(2).results(n)[i], ...);
    v.results(n).push(result_name);
  }
}
\end{code}
\end{samepage}

\FloatBarrier
    
As an~example, we provide a~simple box which performs modulo~4 on integers. We will use nonvectorized store and load instructions and a~vectorized mod4 instruction on 16~bit integers as our instruction set.

\graph{mod4}

\FloatBarrier

\begin{code}
uint_16 a_0 = input_a[i++];
uint_16 a_1 = input_a[i++];
uint_32 a_conversion_w2_0 = a_0 | (a_1 << 16);
uint_32 b = a_conversion_w2_0 & 0x00030003;
uint_16 b_conversion_w1_0 = b & 0xFFFF;
uint_16 b_conversion_w1_1 = (b & 0xFFFF0000) >> 16;
output_c[j++] = b_conversion_w1_0;
output_c[j++] = b_conversion_w1_1;
\end{code}

\begin{rem}
  The condition requiring all widths to be powers of two ensures two things:
  \begin{itemize}
    \item All data vectors are aligned.
    \item We are able to process the same amount of data ($w$) at every instruction.
  \end{itemize}
  Without this condition we would want $w$ to be the least common multiple of the maximum widths of all operations in $G$. 
\end{rem}

