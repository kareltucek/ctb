Consider the~following problem.

\begin{problem}[simple code generation]
Let $G(V,E)$ be a~flow~graph consisting of regular operations. Let $O$ denote some subset of arithmetic operations of~the~C~language. Also, let $G$ be~consistent with the~semantics of these operations. We wish to~generate a~C~code which will~ realize the~graph~$G$.
\end{problem}

\FloatBarrier

Solution to this~problem is straightforward. First, we order vertices~of~$G$ topologically. Then we take vertices in this order and process them one~by~one. For every vertex~$v$ we use its~annotation to transform its~input~data into its~output~data. Then we save the~output~data under some unique identifier for~later~use.

\parspace

We provide graph representation and resulting code of an~adder~box, which simply loads two values, adds them and stores the~result.

\graph{adder}

\begin{code}
int var_a = input_0[i];
int var_b = input_1[i];
int var_c = a + b;
output_0[i] = c;
\end{code}

\FloatBarrier

We may have noticed that vertices are basically of three types, each of some typical form. 

\begin{itemize}
  \item Input -- \emph{``\textless type\textgreater  \textless variable name\textgreater  = \textless load expression\textgreater ;''}
  \item Output -- \emph{``\textless store expression\textgreater ;''}
  \item Operation -- \emph{``\textless type\textgreater  \textless variable name\textgreater  = \textless operation\textgreater (\textless argument 1\textgreater , \textless argument 2\textgreater );''}
\end{itemize}

This allows us to construct simple patterns which may be~used for actual~code~generation. These may~be the~following:

\begin{itemize}
  \item Input -- \emph{``\$type \$name = \$input;''}
  \item Operation -- \emph{``\$type \$name = \$operation;''}
  \item Output -- \emph{``\$output;''}
\end{itemize}

Where $\$input$ and $\$output$ evaluate to environment-dependent expressions, $\$name$ evaluates to a~new~identifier and $\$operation$ evaluates to a~pattern identified~by the~$op$~annotation. The operation pattern may~contain $\$arg1$, $\$arg2$, $\$arg3$... expressions. These evaluate to the~saved~names from incoming vertices.

\parspace

This scheme~of~generation is actually used --- our generator will~be driven by a~quite complex text-processing system, which will be~based on~recursive evaluation of~shell-like variables in~broader contexts provided~by various extensions.

